\documentclass{article}
\usepackage{theorem}

\newcommand{\implies}{\Rightarrow}
\newcommand{\have}{{.~}}
\newcommand{\bstate}{{\mbox{\cal S}}}
\newcommand{\blocks}{{\mbox{\cal B}}}
\newcommand{\tabtop}{{\mbox{\sc t}}}
\newcommand{\tblocks}{{\blocks_\tabtop}}
\newcommand{\bclear}{{\mbox{\sc clear}}}
\newcommand{\bon}{{\mbox{\sc on}}}
\newcommand{\st}{~|}

\begin{document}

\section{Definitions}

\begin{itemize}
\item Let
  $$ \blocks = \{ 1, \ldots, N \} $$
represent a set of $N$ blocks in a blocks-world problem.
We represent the table top with the symbol $\tabtop$, and
define $$ \tblocks \equiv \blocks \cup \{ \tabtop \} $$
  
\item Call any 
  $$ \bstate \subseteq \blocks \times \tblocks $$
a {\em part-state} of the blocks-world problem. The pairs of blocks in
$\bstate$ represent {\em on} relations.

\item A part-state $\bstate$ is {\em legal} iff
  \begin{itemize}
  \item No block is on two things: $$
    \forall i \in \blocks \have
      \langle i , j \rangle \in \bstate \implies
      \not \exists j' \in \tblocks \have
	j \neq j' \wedge \langle i , j' \rangle \in \bstate
  $$
  \item No two blocks are on the same block: $$
    \forall j \in \blocks \have
      \langle i , j \rangle \in \bstate \implies
      \not \exists i' \in \blocks \have
	i \neq i' \wedge \langle i' , j \rangle \in \bstate
  $$
  \end{itemize}

\item A part-state is a {\em total state} or just a {\em state}
iff it is legal and every block is on something: $$
    \forall i \in \blocks \have
      \exists j \in \tblocks \have
	\langle i , j \rangle \in \bstate
$$

\item A legal part-state $\bstate^1$ is {\em simpler} than
a legal part-state $\bstate^2$ iff every block in $\bstate^2$
is either on the same block as on $\bstate^1$ or on the
table: $$
  \forall \langle i, j \rangle \in \bstate^2 \have
    j = \tabtop ~\vee~ \langle i, j \rangle \in \bstate^1
$$
We will justify this terminology in Lemma \ref{simpler-states}
below.
  
\item A block $j$ is {\em clear} in a part-state $\bstate$ iff no block
is above it: $$
  \not \exists i \in \blocks \have \langle i , j \rangle \in \bstate
$$
We will define the set of blocks clear in a part-state $\bstate$ by $$
  \bclear(\bstate) = \{ j \in \bstate \st
  \not \exists i \have \langle i , j \rangle \in \bstate \}
$$

\item A block $i$ is {\em above} a block $i'$ in a
part-state $\bstate$ if there is a sequence
of blocks whose {\em on} relationships lead from $i$ to $i'$: $$
  \exists \{ j_1 , j_2 , \ldots , j_k \} \subseteq \blocks \have
    \{ \langle i , j_1 \rangle , \langle j_1 , j_2 \rangle ,
       \ldots , \langle j_k , i' \rangle \} \subseteq \bstate
$$

\item A part-state $\bstate$ is {\em realizable} iff it
is legal and no block is above itself.

\item {\em Moves} are functions $m_{ij}$ which take a realizable part-state in
which blocks $i$ and $j$ are clear to a
realizable part-state in which $i$ is on $j$.

Let $i \in \blocks$ and $j \in \tblocks$, and let $C(ij)$ be the set
of realizable part-states
in which $i$ is clear, and either $j = \tabtop$ or $j$ is clear.
Then the domain of $m_{ij}$ is
$C(ij)$, and the range is the set of all realizable part-states.

Define the partial functions
$\bon_i(\bstate)$ mapping legal part-states to
blocks such that
$\bon_i(\bstate) = j$ when $\langle i, j \rangle \in \bstate$;
$\bon_i$ is undefined elsewhere.
(In other words, $\bon_i$ says
which block block $i$ is on in a given state.)
By the definition of a legal part-state, these functions are well-defined.
We then define
$m_{ij}$ by $$
  m_{ij}(\bstate) \equiv
    \left ( \bstate - \{ \langle i, \bon_i(\bstate) \rangle \} \right )
    \cup \{ \langle i, j \rangle \}
$$
when $i$ and $j$ are clear in $\bstate$; $m_{ij}$ is undefined
elsewhere.  A move is {\em legal} iff $m_{ij}$ is defined.

\item A {\em move sequence} $M$
of length $k$ from state $\bstate^0$ to state $\bstate^1$ consists of
a sequence of pairs $$
  \langle i_1, j_1 \rangle , \langle i_2 , j_2 \rangle , \ldots ,
  \langle i_k , j_k \rangle \in \left ( \blocks \times \tblocks \right ) ^ \ast
$$
such that $$
  \left ( m_{i_k j_k} \circ \cdots \circ m_{i_2 j_2} \circ
  m_{i_1 j_1} \right ) (\bstate^0) = \bstate^1
$$
We allow ourselves the liberty of writing $M(\bstate^0) = \bstate^1$.
A move sequence $M$ is {\em legal} for a state $\bstate$ iff
$M(\bstate)$ is defined.

\item A {\em primitive-blocks-world} (PBW) {\em problem} consists of two
realizable states: an
initial state $I$ and a goal state $G$.
A {\em solution} to the problem is a move sequence $M$ such that $M(I) = G$.
We say that a solution is {\em optimal} if there is no solution
of shorter length.

\item A {\em stack} is a sequence of blocks $i_1, i_2, \ldots, i_k$
in a realizable
part-state $\bstate$ such that $$
  \{ \langle i_1, i_2 \rangle, \langle i_2, i_3 \rangle, \ldots,
  \langle i_{k-1}, i_k \rangle \} \subseteq \bstate
$$
and such that $$
  \forall i \in \{ i_1, i_2, \ldots, i_{k-1} \} \have
    i \neq \tabtop
$$
A {\em tower} is a stack such that $i_k = \tabtop$.

\end{itemize}

\section{Solution Optimization}

In the search for an algorithm giving optimal solutions for PBW
problems, we make use of the following lemmas, all of which
have been proved previously elsewhere.  All of these lemmas are
proved using essentially the same reasoning:  If $\bstate$ is a
solution, and we can construct from $\bstate$ a solution $\bstate'$ which
is the same as $\bstate$ except for deletion of one or more elements,
then $\bstate$ is not an optimal solution.

\begin{lemma}{simpler-states}{Simpler States}
\given
States $\bstate^1$ and $\bstate^1$ such that $$
\bclear(\bstate^1) \subseteq \bclear(\bstate^2)
$$
A legal move sequence $M$ starting at $\bstate^1$.
\prove
$M$ is also a legal move sequence starting at $\bstate^2$.
\proof
Consider the states just after the first move $\langle i, j \rangle$ in $M$.
\begin{eqnarray*}
  \bstate^1_1 = m_{i j}(\bstate^1) &=&
  \left ( \bstate^1 - \{ \langle i, \bon_i(\bstate^1) \rangle \} \right )
  \cup \{ \langle i, j \rangle \} \\
  \bstate^2_1 = m_{i j}(\bstate^2) &=&
  \left ( \bstate^2 - \{ \langle i, \bon_i(\bstate^2) \rangle \} \right )
  \cup \{ \langle i, j \rangle \}
\end{eqnarray*}
This first move is legal, since it is legal in $\bstate^1$ and
$\bclear(\bstate^1) \subseteq \bclear(\bstate^2)$, and the set of
clear blocks after the move has the property that
\begin{eqnarray*}
  \bclear(\bstate^1_1) &=& \bclear (
  \left ( \bstate^1 - \{ \langle i, \bon_i(\bstate^1) \rangle \} \right )
  \cup \{ \langle i, j \rangle \} ) \\
  &=&  \left ( \bstate^1 - \{ \langle i, \bon_i(\bstate^1) \rangle \} \right )
  - j
\end{eqnarray*}
\end{lemma}

\begin{lemma}{table-to-table}{Table-To-Table Moves}
\given
A PBW problem $\langle I, G \rangle$.
An optimal solution $M$ to $\langle I, G \rangle$.
\prove
No block moves from the table to the table in $M$.
\proof
Assume the contrary of the lemma, i.e., that there is
some table-to-table move in $M$.
Use this move to split $M$ 
$$
  M = M^1 \cdot \langle i, \tabtop \rangle \cdot M^2
$$
where $$
  \bon_i(M^1(I)) = \tabtop
$$
Let $\bstate^1 = M^1(I)$.
Applying the definitions of the previous section yields
\begin{eqnarray*}
  (\langle i, \tabtop \rangle)(\bstate^1)
  &\equiv& m_{i \tabtop}(\bstate^1) \\
  &=& \left ( \bstate^1 - \{ \langle i, \bon_i(\bstate^1) \rangle \} \right )
  \cup \{ \langle i, \tabtop \rangle \} \\
  &=& \left ( \bstate^1 - \{ \langle i, \tabtop \rangle \} \right )
  \cup \{ \langle i, \tabtop \rangle \} \\
  &=& \bstate^1
\end{eqnarray*}
This says that the state at the start of the table-to-table
move is the same as the state at the end, and thus by the
definition of a solution, the sequence obtained by deleting
this move from $M$ is also a solution.  But by the definition
of optimality $M$ is therefore not optimal, and thus we have
a condradiction.

Therefore, no such move can be in $M$.
\end{lemma}


\begin{lemma}{tabled}{Correctly Tabled Blocks}
\given
A PBW problem $\langle I, G \rangle$, such that
a block $i$ is on the table in both $I$ and $G$.
An optimal solution $M$ to $\langle I, G \rangle$.
\prove
$i$ doesn't move in $M$, i.e. $$
  \not \exists j \in \tblocks \have
    M = \ldots, \langle i, j \rangle, \ldots
$$
\proof
Assume the contrary, i.e., that $i$ moves in $M$.

We first split $M$ such that $$
  M = M^1 \cdot \langle i, j \rangle \cdot M^2 \cdot
  \langle i, \tabtop \rangle \cdot M^3
$$
where $M^1$ contains no moves of $i$, and
$M^2$ contains no moves of $i$ to $\tabtop$.
To see that this split is possible, we first note
that since by assumption $i$ moves in $M$, the first
move in the split exists.  Further, by Lemma
\ref{table-to-table}, $j \neq \tabtop$.
Let \begin{eqnarray*}
\bstate^1 &=& M^1(I) \\
\bstate^2 &=& (\langle i, j \rangle)(\bstate^1) \\
\bstate^3 &=& (M^2 \cdot \langle i, \tabtop \rangle)(\bstate^2)
\end{eqnarray*}
Since
$\bon_i(G) = \tabtop$, but $\bon_i(\bstate^2) \neq \tabtop$,
then the second move in the split exists as well.

Now consider the subproblem $\langle \bstate^1, \bstate^3$ and its known
solution $$
  \langle i, j \rangle \cdot M^2 \cdot \langle i, \tabtop \rangle
$$
We will construct a shorter solution to the problem from this
solution.

For any solution $M$ to $\langle I, G \rangle$,
let $M_i$ be the sequence of moves obtained as follows:
Consider any move $m = \langle j, i \rangle$ in $M$, and the
state $\bstate$ just before that move.  Replace $m$ with the two-move
sequence $$
  m' = \langle i, tabtop \rangle, \langle j, \bon_i(\bstate) \rangle
$$
We note that $M_i$ is still a sequence of
legal moves by Lemma \ref{simpler-states}.

\end{lemma}

\end{document}
