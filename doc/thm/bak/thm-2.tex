\documentclass{article}
\usepackage{theorem}

\DeclareMathAlphabet{\mathsc}{OT1}{cmr}{m}{sc}

\newcommand{\implies}{\Rightarrow}
\newcommand{\have}{{.~}}
\newcommand{\bstate}{{\mathcal{S}}}
\newcommand{\blocks}{{\mathcal{B}}}
\newcommand{\tabtop}{{\mathsc{t}}}
\newcommand{\tblocks}{{\blocks_\tabtop}}
\newcommand{\bclear}{{\mathsc{clear}}}
\newcommand{\bon}{{\mathsc{on}}}
\newcommand{\babove}{{\mathsc{above}}}
\newcommand{\bstack}{{\mathsc{stack}}}
\newcommand{\bbase}{{\mathsc{base}}}
\newcommand{\st}{~|}

\begin{document}

\section{Definitions}

\begin{itemize}
\item Let
  $$ \blocks = \{ 1, \ldots, N \} $$
represent a set of $N$ blocks in a blocks-world problem.
We represent the table top with the symbol $\tabtop$, and
define $$ \tblocks \equiv \blocks \cup \{ \tabtop \} $$
  
\item Call any 
  $$ \bstate \subseteq \blocks \times \tblocks $$
a {\em part-state} of the blocks-world problem. The pairs of blocks in
$\bstate$ represent {\em on} relations.

\item A part-state $\bstate$ is {\em legal} iff
  \begin{itemize}
  \item No block is on two things. $$
    \forall i \in \blocks \have
      \langle i , j \rangle \in \bstate \implies
      \not\exists j' \in \tblocks \have
	j \neq j' \wedge \langle i , j' \rangle \in \bstate
  $$
  \item No two blocks are on the same block. $$
    \forall j \in \blocks \have
      \langle i , j \rangle \in \bstate \implies
      \not\exists i' \in \blocks \have
	i \neq i' \wedge \langle i' , j \rangle \in \bstate
  $$
  \end{itemize}

\item Define the partial functions
$\bon_i(\bstate)$ mapping legal part-states to
blocks such that
$\bon_i(\bstate) = j$ when $\langle i, j \rangle \in \bstate$;
$\bon_i$ is undefined elsewhere.
(In other words, $\bon_i$ says
which block block $i$ is on in a given state.)
By the definition of a legal part-state, these functions are well-defined.

\item A part-state is a {\em total state} or just a {\em state}
iff it is legal and every block is on something. $$
    \forall i \in \blocks \have
      \exists j \in \tblocks \have
	\langle i , j \rangle \in \bstate
$$

\item The {\em above} relation is the transitive closure of
the {\em on} relation: A block $i$ is above a block $j$ in a
legal part-state $\bstate$ if either $i$ is on $j$, or $i$ is on
a block which is above $j$.  $$
  \babove_i(\bstate) = \left \{ \begin{array}{ll}
    \{ j \} \cup \babove_j(\bstate) &
      j = \bon_i(\bstate) \wedge j \neq \tabtop \\
    \emptyset & \mbox{otherwise}
  \end{array} \right .
$$

\item A block $j$ is {\em clear} in a legal part-state $\bstate$ iff no block
is above it. $$
  \bclear(\bstate) = \{ j \in \bstate \st
  \not\exists i \have \langle i , j \rangle \in \bstate \}
$$

\item A legal part-state $\bstate$ is {\em realizable} iff
no block is above itself. $$
  \not\exists i \in \blocks \have
    i \in \babove_i(\bstate)
$$

\item {\em Moves} are functions $m_{ij}$ which take a realizable part-state in
which blocks $i$ and $j$ are clear to a
realizable part-state in which $i$ is on $j$.

Let $i \in \blocks$ and $j \in \tblocks$, and let $\mathcal{C}_{ij}$ be the set
of realizable part-states
in which $i$ is clear, and either $j = \tabtop$ or $j$ is clear.
Then the domain of $m_{ij}$ is
$\mathcal{C}_{ij}$, and the range is the set of all realizable part-states.

We define the partial function $m_{ij}$ by $$
  m_{ij}(\bstate) \equiv
    \left ( \bstate - \{ \langle i, \bon_i(\bstate) \rangle \} \right )
    \cup \{ \langle i, j \rangle \}
$$
when $i$ and $j$ are clear in $\bstate$; $m_{ij}$ is undefined
elsewhere.  A move is {\em legal} iff $m_{ij}$ is defined.

\item A {\em move sequence} $M$
of length $k$ from state $\bstate^1$ to state $\bstate^2$ consists of
a sequence of pairs $$
  \langle i_1, j_1 \rangle , \langle i_2 , j_2 \rangle , \ldots ,
  \langle i_k , j_k \rangle \in \left ( \blocks \times \tblocks \right ) ^ \ast
$$
such that $$
  \left ( m_{i_k j_k} \circ \cdots \circ m_{i_2 j_2} \circ
  m_{i_1 j_1} \right ) (\bstate^1) = \bstate^2
$$
We allow ourselves the liberty of writing $M(\bstate^1) = \bstate^2$.
A move sequence $M$ is {\em legal} for a state $\bstate$ iff
$M(\bstate)$ is defined.

\item A {\em primitive-blocks-world} (PBW) {\em problem} consists of two
realizable states: an
initial state $I$ and a goal state $G$.
A {\em solution} to the problem is a move sequence $M$ such that $M(I) = G$.
We say that a solution is {\em optimal} if there is no solution
of shorter length.

\item A {\em stack} of blocks starting at $i$ in a
legal part-state $\bstate$ is just
$i$ plus the set of blocks $i$ is above. $$
  \bstack_i(\bstate) = \{ i \} \cup \babove_i(\bstate)
$$
We call $i$ the {\em top} of the stack.
If $\bstate$ is a realizable part-state, the {\em base}
of the stack is the block in the stack which is
above no other blocks in the stack. $$
  \bbase_i(\bstate) = j \in \bstack_i(\bstate) \have
    \babove_j(\bstate) \subseteq \{ \tabtop \}
$$
(The lemma that every stack in any realizable part state
has a base is proved by tracing through the definitions.)

\item A {\em tower} is a stack which is a subset of no other
stack.  

\end{itemize}

\section{Flexibility}

A legal part-state $\bstate^2$ is {\em as flexible} as
a legal part-state $\bstate^1$ iff no block in $\bstate^2$
is above some block in $\bstate^2$ which it is not above in
$\bstate^1$, i.e. $$
  \babove_i (\bstate^2) \subseteq \babove_i(\bstate^1)
$$
We notate this by writing $\bstate^2 \sqsubseteq \bstate^1$.

An observation about states related by flexibility is in order.
\begin{lemma}{flexible-clear}{Flexibility And Clear Blocks}
\given
States $\bstate^1$ and $\bstate^2$ with $\bstate^2 \sqsubseteq \bstate^1$.
\prove
Any block clear in $\bstate^1$ is clear in $\bstate^2$: $$
  \bclear(\bstate^2) \supseteq \bclear(\bstate^1)
$$
\proof
Consider some block $j$ in $\bstate^2$.  If it is not clear,
then there is a block $i$ above it in $\bstate^2$.  But,
by definition of flexibility, this means that $i$ is above $j$
in $\bstate^1$, so $j$ is not clear in $\bstate^1$, either.
\end{lemma}

We now justify the terminology ``as flexible as''
by means of Lemma \ref{flexible-move}:
\begin{lemma}{flexible-move}{Flexibility And Move Sequences}
\given
States $\bstate^1$ and $\bstate^2$ with $\bstate^2 \sqsubseteq \bstate^1$.
A legal move sequence $M$ starting at $S^1$.
\prove
\begin{description}
\item[i)~] $M$ is also a legal move sequence starting at $\bstate^2$.
\item[ii)~] $M(\bstate^2) \sqsubseteq M(\bstate^1)$.
\end{description}
\proof
By induction on prefixes of $M$.
\basecase
The theorem is true for the empty prefix of $M$.
\indcase
Decompose $M$ such that the prefix $M^1$ before
some move $m = \langle i, j \rangle$ satisfies the lemma $$
  M = M^1 \cdot \langle i, j \rangle \cdot M^2
$$
and let
\begin{eqnarray*}
  \bstate^1_1 &=& M^1(\bstate^1) \\
  \bstate^1_2 &=& \langle i, j \rangle(\bstate^1_1) \\
  \bstate^2_1 &=& M^1(\bstate^2) \\
  \bstate^2_2 &=& \langle i, j \rangle(\bstate^2_1)
\end{eqnarray*}

By hypothesis we have $\bstate^2_1 \sqsubseteq \bstate^1_1$.
By Lemma \ref{flexible-clear}, this means that $$
  \bclear(\bstate^1_1) \subseteq \bclear(\bstate^2_1)
$$
This means that since $\langle i, j \rangle$ is a legal
move in $\bstate^1_1$, it must also be a legal move in
$\bstate^2_1$.  Thus, condition (i) of the lemma holds
for the prefix $M^1 \cdot \langle i, j \rangle$ of $M$.

To see that condition (ii) of the Lemma holds on the extended
prefix, we note that the only block possibly above other blocks in
$\bstate^2_2$ but not in $\bstate^2_1$ is $i$.  We then note that
by definition of flexibility
\begin{eqnarray*}
  \babove_i(\bstate^2_2) &=& \babove_j(\bstate^2_1) \cup \{j\} \\
  &\subseteq& \babove_j(\bstate^1_1) \cup \{j\} \\
  &=& \babove_i(\bstate^1_2)
\end{eqnarray*}
\end{lemma}
  
\section{Solution Optimization}

In the search for an algorithm giving optimal solutions for PBW
problems, we make use of the following lemmas, most of which
have been proved previously elsewhere in the literature.  All
of these lemmas are proved using essentially the same
reasoning:  If $\bstate$ is a solution, and we can construct
from $\bstate$ a solution $\bstate'$ which is the same as
$\bstate$ except for deletion of one or more elements, then
$\bstate$ is not an optimal solution.

\begin{lemma}{table-to-table}{Table-To-Table Moves}
\given
A PBW problem $\langle I, G \rangle$.
An optimal solution $M$ to $\langle I, G \rangle$.
\prove
No block moves from the table to the table in $M$.
\proof
Assume the contrary of the lemma, i.e., that there is
some table-to-table move in $M$.
Use this move to split $M$ 
$$
  M = M^1 \cdot \langle i, \tabtop \rangle \cdot M^2
$$
where $$
  \bon_i(M^1(I)) = \tabtop
$$
Let $\bstate^1 = M^1(I)$.
Applying the definitions of the previous section yields
\begin{eqnarray*}
  \langle i, \tabtop \rangle(\bstate^1)
  &\equiv& m_{i \tabtop}(\bstate^1) \\
  &=& \left ( \bstate^1 - \{ \langle i, \bon_i(\bstate^1) \rangle \} \right )
  \cup \{ \langle i, \tabtop \rangle \} \\
  &=& \left ( \bstate^1 - \{ \langle i, \tabtop \rangle \} \right )
  \cup \{ \langle i, \tabtop \rangle \} \\
  &=& \bstate^1
\end{eqnarray*}
This says that the state at the start of the table-to-table
move is the same as the state at the end, and thus by the
definition of a solution, the sequence obtained by deleting
this move from $M$ is also a solution.  But by the definition
of optimality $M$ is therefore not optimal, and thus we have
a condradiction.

Therefore, no such move can be in $M$.
\end{lemma}


\begin{lemma}{tabled}{Correctly Tabled Blocks}
\given
A PBW problem $\langle I, G \rangle$, such that
a block $i$ is on the table in both $I$ and $G$.
An optimal solution $M$ to $\langle I, G \rangle$.
\prove
$i$ doesn't move in $M$, i.e. $$
  \not\exists j \in \tblocks \have
    M = \ldots, \langle i, j \rangle, \ldots
$$
\proof
Assume the contrary, i.e., that $i$ moves in $M$.

We first split $M$ such that $$
  M = M^1 \cdot \langle i, j \rangle \cdot M^2 \cdot
  \langle i, j' \rangle \cdot M^3
$$
where $M^2$ and $M^3$ contain no moves of $i$.
To see that this split is possible, let
\begin{eqnarray*}
\bstate^1 &=& M^1(I) \\
\bstate^2 &=& \langle i, j \rangle(\bstate^1) \\
\bstate^3 &=& (M^2 \cdot \langle i, \tabtop \rangle)(\bstate^2)
\end{eqnarray*}
We first note
that since by assumption $i$ moves in $M$, the first
move in the split exists.  Further, by Lemma
\ref{table-to-table}, $j \neq \tabtop$.
Finally, since $i$ is on the table in $G$, and
$j$ is not the table in $\bstate^2$, then $i$ must move
to the table, so the second move in the split exists, and $j' = \tabtop$.

Now consider the subproblem $\langle \bstate^1, \bstate^3 \rangle$
and its known and presumably optimal solution $$
  M' = \langle i, j \rangle \cdot M^2 \cdot \langle i, \tabtop \rangle
$$
We will construct a shorter solution to the problem from this
solution.

For any solution $M$ to $\langle I, G \rangle$,
let $M_i$ be the sequence of moves obtained as follows:
Consider any move $m = \langle j, i \rangle$ in $M$, and the
state $\bstate$ just before that move.  Replace $m$ with the two-move
sequence $$
  m' = \langle i, tabtop \rangle, \langle j, \bon_i(\bstate) \rangle
$$
We note that $M_i$ is still a sequence of
legal moves by Lemma \ref{flexible-move}, since the state after $m'$
is as flexible as the state after $m$.

Now consider $M'_i$.  Since $i$ doesn't move $M^2$, the last
move in $M'_i$ moves $i$ from the table to the table, and can
thus be deleted.  And since the only $\babove$ relations that
have changed anywhere in the two move sequences involve $i$,
and since $i$ is on the table at the end of both $M'_i$ and $M_i$,
$M'_i(\bstate) = M_i(\bstate)$.

Thus, we now have a sequence 
\end{lemma}

\end{document}
