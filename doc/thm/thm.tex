\documentclass{article}
\usepackage{theorem}

\DeclareMathAlphabet{\mathsc}{OT1}{cmr}{m}{sc}

\newcommand{\implies}{\Rightarrow}
\newcommand{\have}{{.~}}
\newcommand{\bstate}{{\mathcal{S}}}
\newcommand{\blocks}{{\mathcal{B}}}
\newcommand{\tabtop}{{\mathsc{t}}}
\newcommand{\tblocks}{{\blocks_\tabtop}}
\newcommand{\bclear}{{\mathsc{clear}}}
\newcommand{\bon}{{\mathsc{on}}}
\newcommand{\babove}{{\mathsc{above}}}
\newcommand{\bstack}{{\mathsc{stack}}}
\newcommand{\bbase}{{\mathsc{base}}}
\newcommand{\bflexed}{{\mathsc{flexed}}}
\newcommand{\st}{~|}

\begin{document}

\section{Definitions}

\begin{itemize}
\item Let
  $$ \blocks = \{ 1, \ldots, N \} $$
represent a set of $N$ blocks in a blocks-world problem.
We represent the table top with the symbol $\tabtop$, and
define $$ \tblocks \equiv \blocks \cup \{ \tabtop \} $$
  
\item Call any 
  $$ \bstate \subseteq \blocks \times \tblocks $$
a {\em part-state} of the blocks-world problem. The pairs of blocks in
$\bstate$ represent {\em on} relations.

\item A part-state $\bstate$ is {\em legal} iff
  \begin{itemize}
  \item No block is on two things. $$
    \forall i \in \blocks \have
      \langle i , j \rangle \in \bstate \implies
      \not\exists j' \in \tblocks \have
	j \neq j' \wedge \langle i , j' \rangle \in \bstate
  $$
  \item No two blocks are on the same block. $$
    \forall j \in \blocks \have
      \langle i , j \rangle \in \bstate \implies
      \not\exists i' \in \blocks \have
	i \neq i' \wedge \langle i' , j \rangle \in \bstate
  $$
  \end{itemize}

\item Define the partial functions
$\bon_i(\bstate)$ mapping legal part-states to
blocks such that
$\bon_i(\bstate) = j$ when $\langle i, j \rangle \in \bstate$;
$\bon_i$ is undefined elsewhere.
(In other words, $\bon_i$ says
which block block $i$ is on in a given state.)
By the definition of a legal part-state, these functions are well-defined.

\item A part-state is a {\em total state} or just a {\em state}
iff it is legal and every block is on something. $$
    \forall i \in \blocks \have
      \exists j \in \tblocks \have
	\langle i , j \rangle \in \bstate
$$

\item The {\em above} relation is the transitive closure of
the {\em on} relation: A block $i$ is above a block $j$ in a
legal part-state $\bstate$ if either $i$ is on $j$, or $i$ is on
a block which is above $j$.  $$
  \babove_i(\bstate) = \left \{ \begin{array}{ll}
    \{ j \} \cup \babove_j(\bstate) &
      j = \bon_i(\bstate) \wedge j \neq \tabtop \\
    \emptyset & \mbox{otherwise}
  \end{array} \right .
$$

\item A block $j$ is {\em clear} in a legal part-state $\bstate$ iff no block
is above it. $$
  \bclear(\bstate) = \{ j \in \bstate \st
  \not\exists i \have \langle i , j \rangle \in \bstate \}
$$

\item A legal part-state $\bstate$ is {\em realizable} iff
no block is above itself. $$
  \not\exists i \in \blocks \have
    i \in \babove_i(\bstate)
$$

\item {\em Moves} are functions $m_{ij}$ which take a realizable part-state in
which blocks $i$ and $j$ are clear to a
realizable part-state in which $i$ is on $j$.

Let $i \in \blocks$ and $j \in \tblocks$, and let $\mathcal{C}_{ij}$ be the set
of realizable part-states
in which $i$ is clear, and either $j = \tabtop$ or $j$ is clear.
Then the domain of $m_{ij}$ is
$\mathcal{C}_{ij}$, and the range is the set of all realizable part-states.

We define the partial function $m_{ij}$ by $$
  m_{ij}(\bstate) \equiv
    \left ( \bstate - \{ \langle i, \bon_i(\bstate) \rangle \} \right )
    \cup \{ \langle i, j \rangle \}
$$
when $i$ and $j$ are clear in $\bstate$; $m_{ij}$ is undefined
elsewhere.  A move is {\em legal} iff $m_{ij}$ is defined.

\item A {\em move sequence} $M$
of length $k$ from state $\bstate^1$ to state $\bstate^2$ consists of
a sequence of pairs $$
  \langle i_1, j_1 \rangle , \langle i_2 , j_2 \rangle , \ldots ,
  \langle i_k , j_k \rangle \in \left ( \blocks \times \tblocks \right ) ^ \ast
$$
such that $$
  \left ( m_{i_k j_k} \circ \cdots \circ m_{i_2 j_2} \circ
  m_{i_1 j_1} \right ) (\bstate^1) = \bstate^2
$$
We allow ourselves the liberty of writing $M(\bstate^1) = \bstate^2$.
A move sequence $M$ is {\em legal} for a state $\bstate$ iff
$M(\bstate)$ is defined.

\item A {\em primitive-blocks-world} (PBW) {\em problem} consists of two
realizable states: an
initial state $I$ and a goal state $G$.
A {\em solution} to the problem is a move sequence $M$ such that $M(I) = G$.
We say that a solution is {\em optimal} if there is no solution
of shorter length.

\item A {\em stack} of blocks starting at $i$ in a
legal part-state $\bstate$ is just
$i$ plus the set of blocks $i$ is above. $$
  \bstack_i(\bstate) = \{ i \} \cup \babove_i(\bstate)
$$
We call $i$ the {\em top} of the stack.
If $\bstate$ is a realizable part-state, the {\em base}
of the stack is the block in the stack which is
above no other blocks in the stack. $$
  \bbase_i(\bstate) = j \in \bstack_i(\bstate) \have
    \babove_j(\bstate) \subseteq \{ \tabtop \}
$$
(The lemma that every stack in any realizable part state
has a base is proved by tracing through the definitions.)

\item A {\em tower} is a stack which is a subset of no other
stack.  

\end{itemize}

\section{Basics}

We first prove that two states are the same exactly when
all the blocks have the same $\babove$ relation.
\begin{lemma}{same-above}{$\babove$ As A State Characterization}
\given
States $\bstate^1$ and $\bstate^2$.
\prove
$\bstate^1 = \bstate^2$ iff $$
  \forall i \in \blocks \have
    \babove_i(\bstate^1) = \babove_i(\bstate^2)
$$
\proof  By forward and backward implication.

\forwardcase
By definition, $\babove$ is a function.  Thus, 
$\babove_i$ applied to any identical pair of states
must yield identical answers.

\backwardcase
We prove this case by contradiction.
Assume that $\bstate^1 = \bstate^2$, but there is some $i$ such that
$\babove_i(\bstate^1) \not= \babove_i(\bstate^2)$.
WLOG, assume that 
there is some $j \in \tblocks$
such that $j \in \babove_i(\bstate^1)$ but not $j \in \babove_i(\bstate^2)$.
Then by definition of $\babove$, there must exist some
$k \in i \cup \babove_i(\bstate^1)$
such that $\bon_k(\bstate^1) = j$ but $\bon_k(\bstate^2) \not= j$.
But by definition of $\bon$, this implies that $\bstate^1 \not= \bstate^2$,
contradicting our original assumption.
\end{lemma}

\section{Flexibility}

A legal part-state $\bstate^2$ is {\em as flexible} as
a legal part-state $\bstate^1$ iff no block in $\bstate^2$
is above some block in $\bstate^2$ which it is not above in
$\bstate^1$, i.e. $$
  \babove_i (\bstate^2) \subseteq \babove_i(\bstate^1)
$$
We notate this by writing $\bstate^2 \sqsubseteq \bstate^1$.

An observation about states related by flexibility is in order.
\begin{lemma}{flexible-clear}{Flexibility And Clear Blocks}
\given
States $\bstate^1$ and $\bstate^2$ with $\bstate^2 \sqsubseteq \bstate^1$.
\prove
Any block clear in $\bstate^1$ is clear in $\bstate^2$: $$
  \bclear(\bstate^2) \supseteq \bclear(\bstate^1)
$$
\proof
Consider some block $j$ in $\bstate^2$.  If it is not clear,
then there is a block $i$ above it in $\bstate^2$.  But,
by definition of flexibility, this means that $i$ is above $j$
in $\bstate^1$, so $j$ is not clear in $\bstate^1$, either.
\end{lemma}

We now justify the terminology ``as flexible as''
by means of Theorem \ref{flexible-move}:
\begin{theorem}{flexible-move}{Flexibility And Move Sequences}
\given
States $\bstate^1$ and $\bstate^2$ with $\bstate^2 \sqsubseteq \bstate^1$.
A legal move sequence $M$ starting at $\bstate^1$.
\prove
\begin{description}
\item[i)~] $M$ is also a legal move sequence starting at $\bstate^2$.
\item[ii)~] $M(\bstate^2) \sqsubseteq M(\bstate^1)$.
\end{description}
\proof
By induction on prefixes of $M$.

\basecase
The theorem is true for the empty prefix of $M$.

\indcase
Decompose $M$ such that the prefix $M^1$ before
some move $m = \langle i, j \rangle$ satisfies the lemma $$
  M = M^1 \cdot \langle i, j \rangle \cdot M^2
$$
and let
\begin{eqnarray*}
  \bstate^1_1 &=& M^1(\bstate^1) \\
  \bstate^1_2 &=& \langle i, j \rangle(\bstate^1_1) \\
  \bstate^2_1 &=& M^1(\bstate^2) \\
  \bstate^2_2 &=& \langle i, j \rangle(\bstate^2_1)
\end{eqnarray*}

By hypothesis we have $\bstate^2_1 \sqsubseteq \bstate^1_1$.
By Lemma \ref{flexible-clear}, this means that $$
  \bclear(\bstate^1_1) \subseteq \bclear(\bstate^2_1)
$$
This means that since $\langle i, j \rangle$ is a legal
move in $\bstate^1_1$, it must also be a legal move in
$\bstate^2_1$.  Thus, condition (i) of the lemma holds
for the prefix $M^1 \cdot \langle i, j \rangle$ of $M$.

To see that condition (ii) of the Lemma holds on the extended
prefix, we note that the only block possibly above other blocks in
$\bstate^2_2$ but not in $\bstate^2_1$ is $i$.  We then note that
by definition of flexibility
\begin{eqnarray*}
  \babove_i(\bstate^2_2) &=& \babove_j(\bstate^2_1) \cup \{j\} \\
  &\subseteq& \babove_j(\bstate^1_1) \cup \{j\} \\
  &=& \babove_i(\bstate^1_2)
\end{eqnarray*}
\end{theorem}

If two states are related by flexibility, legal sequences of
moves tend to make the two states the same.  The next theorem
quantifies this observation.

Given a block $i$ and states $\bstate$ and $\bstate'$ such
that $\bstate' \sqsubseteq \bstate$, we will say that block
$i$ is {\em flexed} iff $i$ is above fewer blocks in
$\bstate'$ than in $bstate$. $$
  \babove_i(\bstate') \subset \babove_i(\bstate)
$$
We will write $\bflexed_{\bstate}(\bstate')$ for the
set of blocks in $\bstate'$ flexed with respect to $\bstate$.

\begin{theorem}{flexed-blocks}{Flexed Blocks}
\given States $\bstate^1_1$ and $\bstate^2_1$ such that
$\bstate^1_1 \sqsubseteq \bstate^2_1$.  A
move sequence $M$ such that $M(\bstate^1_1) = \bstate^1_2$
and $M(\bstate^2_1) = M(\bstate^2_2)$.
\prove
The final states are the same whenever all the flexed blocks
in the initial state have moved, i.e.,
$\bstate^2_1 = \bstate^2_2$ iff $$
  \forall i \in \bflexed_{\bstate^2_1}(\bstate^1_1) \have
    \exists j \in \tblocks \have
      \langle i, j \rangle \in M
$$
\proof  By forward and backward implication.

\forwardcase
We first prove that when the final states are the same, all
flexed blocks have moved.
By Lemma \ref{same-above} two states are the same exactly when the
$\babove$ relations of all of their blocks are the same.
Now consider any block $i$ 
which is flexed in a state $\bstate^1$ WRT a state $\bstate^1_1$. $$
  i \in \bflexed_{\bstate^1_1}(\bstate^1)
$$
Assume that a sequence $M$ of moves takes both
initial states $\bstate^1$ and $\bstate^1_1$
to the same final state $\bstate^2$. $$
  \bstate^2 = M(\bstate^1) = M(\bstate^1_1)
$$
By the definition of flexed
block, the $\babove$ relation of $i$ in the
more flexible initial state $\bstate^1$ was a subset
of that in the less flexible initial state $\bstate^1_1$. $$
   \babove_i(\bstate^1) \subset \babove_i(\bstate^1_1)
$$
Since the final
states are the same at the end of the sequence of moves, then
the sequence of moves must have changed the $\babove_i$ relation
WRT at least one of the initial states $\bstate^1$ and $\bstate^1_1$.
But, by definition of $\babove_i$, this can happen only if block $i$ moves.

\backwardcase
We next prove that when all flexed blocks have moved, the states are
identical.  We prove this case by induction.

\end{theorem}
  
\section{Solution Optimization}

\end{document}
